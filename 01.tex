\documentclass[ignorenonframetext,hyperref={pdftex,unicode}]{beamer}

\usetheme{Esperanto}

\title{Первая встреча Эсперанто клуба}
\author[Эсперанто клуб Klika Tech]{Эсперанто клуб Klika Tech\\ andrej@zahar.ws}


\begin{document}

\frame{\titlepage}

\begin{frame}{Преамбула}
    \begin{itemize}
	    \item Эсперанто - фонетический язык. Т.е. записывается именно так, как потом звучит. Как пишутся и звучат буквы в эсперанто можно посмотреть тут: \href{https://lernu.net/ru/gramatiko/skribo}{https://lernu.net/ru/gramatiko/skribo}
        \item Если у вас нет правильной раскладки для эсперанто, то буквы вроде ŝ, ĝ, ŭ, ĵ, ĥ и ĉ можно заменять сочетаниями с буквами, которых нет в эсперанто. Например: sx, gx, ux, jx, hx и cx.
        \item Если в слове два и более слога — ударение всегда только на предпоследний. Без исключений!
    \end{itemize}
\end{frame}

\begin{frame}{Существительные и прилагательные}
    Cуществительные заканчиваются на -o, прилагательные на -a. Примеры:
    \begin{itemize}
        \item viro - мужчина
        \item knabo - мальчик
        \item hundo - собака
        \item kato - кот
    \end{itemize}
    \begin{itemize}
        \item bona - хороший
        \item feliĉa - счастливый
        \item rapida - быстрый
        \item granda - большой
        \item juna - молодой
    \end{itemize}
\end{frame}

\begin{frame}{Женский род}
    Cуществительный женского рода образуются добавлением суффикса \textbf{-in-}:
    \begin{itemize}
        \item virino - женщина
        \item knabino - девочка
        \item hundino - собака женского рода
        \item katino - кошка
    \end{itemize}
\end{frame}

\begin{frame}{Множественное число}
    Множественное число образуется добавлением к существительному (и относящимся к нему прилагательным) окончания \textbf{-j}:
    \begin{itemize}
        \item virinoj - женщины
        \item knabinoj - девочки
        \item hundinoj - собаки
        \item katinoj - кошки
        \item grandaj hunoj - большие собаки
        \item junaj virinoj - молодые женщины
    \end{itemize}
\end{frame}

\begin{frame}{Омонимы или волшебное удвоение словаря прилагательных}
    Добавив приставку \textbf{mal-} можно получить прилагательное-антоним (с противоположным значением):
    \begin{itemize}
        \item malbona - плохой
        \item malfeliĉa - несчастный
        \item malrapida - медленный
        \item malgranda - маленький
        \item maljuna - старый
    \end{itemize}
\end{frame}

\begin{frame}
    Неопределённый артикль - просто отсутствие артикля. Определённый арктикль - \textbf{La}.
    \begin{itemize}
        \item Malfeliĉa viro - какой-то несчастный человек (абстрактный, например, или неизвестный собеседникам). 
        \item La bona knabino - определённая или известная собеседникам хорошая девочка.
    \end{itemize}
\end{frame}

\begin{frame}{Глаголы}
    Глаголы имеют окончания:
    \begin{itemize}
        \item<+-> \textbf{-i} - инфинитив. Словари и очень специальные случаи.
        \item<+-> \textbf{-as} - настоящее время.
        \item<+-> \textbf{-is} - прошедшее время.
    \end{itemize}
    \onslide<+-> Глагол "to be" не опускается, как в русском. \textbf{esti}-\textbf{estas}-\textbf{estis}
    \begin{itemize}
        \item<+-> La viro estas rapida - Мужчина быстр.
        \item<+-> La virino estis juna - женщина была молода.
    \end{itemize}
\end{frame}

\begin{frame}{Союзы}
    \begin{itemize}
        \item kaj - и
        \item aŭ - или
        \item sed - но
        \item ĉar - потому что
    \end{itemize}
\end{frame}

\begin{frame}{Просто полезные слова}
    \begin{itemize}
        \item Saluton! - Привет!
        \item Mi petas - пожалуйста
        \item Ĝis revido - до встречи (до свиданья)
        \item Dankon - спасибо
    \end{itemize}
\end{frame}
 \end{document}
